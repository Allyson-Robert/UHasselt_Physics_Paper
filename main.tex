\documentclass{uhphyspaper} 
\usepackage{lipsum} % just for the demo, may be removed

% Titles
\title{Title (bold)*}

\runningtitle{*Running title: \textit{Short title ($\mathit{< 50}$ characters inc. spaces, italic, leave empty if not needed)}}
\bachyear{$\mathit{3^{e}}$}
\course{Bachelorproef}
\date{\today}

% Authors and affiliations
\author[1]{Student Name} % First author
\author[2]{Supervisor name} % Supervisors
\author[3]{Second supervisor}
\author[4*]{Principle Investigator} % Promotor
\affil[1]{UHasselt, Agoralaan, 3590 Diepenbeek, Belgium}
\affil[2, 3]{Specify Group}
\affil[1,4*]{UHasselt, ResearchGroup, Agoralaan, 3590 Diepenbeek, Belgium}

% PI contact info
\pitel{+32 (11) 11 11 11}
\pimail{PImail@uhasselt.be}


% START DOCUMENT
\begin{document}
\maketitle\thispagestyle{fancy} % pagestyle ensures header is present

\begin{multicols}{2}

\section*{ABSTRACT}
\textbf{Abstract tekst mag hier vetgedrukt, $\mathbf{< 250}$ woorden. 
Hier is voorbeeldtekst: \lipsum[1-1]}
\newline
\rule[1.5ex]{\linewidth}{0.4pt}

\section{INTRODUCTION}
Leid je project in, background, doel \& onderzoeksvragen, \dots 
Zorg voor voldoende referenties zoals \cite{knuth:1984} \cite{latex2e}(Endnote/Mendeley/Zotero).

\section{THEORY (optional)}
Tussentitels zijn aangeraden per onderdeel (2.1, 2.2, \dots).
Zijn er essentiele stukken theorie die noodzakelijk zijn om je verslag te begrijpen?

\section{EXPERIMENTAL PROCEDURES}
Tussentitels per techniek (3.1, 3.2, \dots).
Welke technieken zijn er gebruikt?
Zorg dat je onderzoek repliceerbaar is (maar zonder overbodige details).

\section{RESULTS}
Tussentitels per topic/techniek (4.1, 4.1.1, \dots).
Beschrijf de resultaten (interpreteer ze nog niet), wat zie je (objectief)?
Zorg dat figuren een caption hebben (incl. alle afkortingen) onder de figuur en dat je zeker in de tekst refereert naar desbetreffende figuur. 
Tabellen krijgen een caption bovenaan (zelfde principe).

\section{DISCUSSION}
Tussentitels aangeraden.
Beschrijf welke conclusies er getrokken kunnen worden uit je resultaten. 
Leg linken (zowel tussen je eigen resultaten als dingen die je vindt in de literatuur).
\npar
Ook hier is extra voorbeeldtext: \lipsum[1-1]

\section{CONCLUSION}
Kort en bondig, wat leren we hier uit? 
Wat is de outlook?

\rule{\linewidth}{0.4pt}
\textit{Author contributions} -- \textbf{Who did what?} 
JH and JdR conceived and designed the research. 
JH and BvH performed experiments and data analysis. 
EvS and DD provided assistance with FLIM-FRET. JH and JdR wrote the paper. 
All authors carefully edited the manuscript.

\textit{Acknowledgements} -- \textbf{Who helped you? Who do you want to thank.} 
F.E. BvH acknowledges the agency for Innovation by Science and Technology (IWT Flanders) for his doctoral fellowship. 
JH is grateful for a postdoctoral scholarship from the Research Foundation Flanders (FWO Vlaanderen). 
Prof. Em. Yves Engelborghs is thanked for providing access to the confocal microscope. 
Research was funded by grants from the BOF KU Leuven, FWO, and EU (FP7 CHAARM).

\end{multicols}

\vspace{0.5cm}
\printbibliography[title={REFERENCES}]
\newpage


\end{document}
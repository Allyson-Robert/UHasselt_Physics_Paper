\section{THEORY (optional)}
% Edit below
Tussentitels zijn aangeraden per onderdeel (2.1, 2.2, \dots).
Zijn er essentiele stukken theorie die noodzakelijk zijn om je verslag te begrijpen?

Vergelijkingen doe je typisch met de \textit{align} environment.
Dit laat je toe om \& (en \&\&) te gebruiken om alles mooi uit te lijnen.
Wil je midden in een regel een vergelijking gebruiken doe het dan als volgt: $S = k\ln(W)$.

    \begin{align}
    x \times x   &= (x \times x) + 0  &&\textit{ Neutr. El. }+\\
            &= (x \times x) + (x \times \neg x) && \textit{ Compl. }\times \\
            &= x + (x \times \neg x) && \textit{ Distr. }+ \\
            &= x + 0 && \textit{ Compl. }\times\\
            &= x && \textit{ Neutr. El. }+ \label{eq:ne}
    \end{align}

Zorg steeds voor een witregel voor en na een environment wanneer je deze gebruikt. 
Deze witregels zorgen ervoor dat ze beschouwd worden als paragraaf op zich en zo kan \LaTeX de juiste spati\"ering voorzien.
Bovendien blijft je bestand op deze manier ook beter leesbaar.
